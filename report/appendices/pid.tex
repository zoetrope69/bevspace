\chapter{Project Initiation Document}

\section{Basic Details}

\begin{table}[h]
\centering
\begin{tabular}{|
>{\columncolor[HTML]{EFEFEF}}l |l|}
\hline
{\color[HTML]{333333} Student Name}        & Zac Colley                                   \\ \hline
{\color[HTML]{333333} Draft project title} & Free Automated Interconnected Beverage Brewing \\ \hline
{\color[HTML]{333333} Course}              & Web Technologies                                \\ \hline
{\color[HTML]{333333} Client organisation} & GetBrewing.uk                                             \\ \hline
Client contact name                        & Alan Thompson                                             \\ \hline
Project supervisor                         & Rich Boakes                                     \\ \hline
\end{tabular}
\end{table}

\section{Outline of the project environment and problem to be solved}

\textit{In this project I aim to develop a web-based system that will support home-brewers and help to ensure a successful and repeatable brewing process. The system may integrate with brewing hardware and web services in order to provide record and act upon real-time brewing data.}

The project is in partnership with Alan Thompson from GetBrewing.uk, a home brewing company on Elm Grove in Portsmouth.

One of the his products, the Grainfather as a simple control and sensor system. It currently has: a temperature sensor with an LED screen to display this temperature, manual control of the pump and manual control of the heating element.

\textbf{He wishes to offer his brewing kit with an improved control system.}

The proposed solution would improve the Grainfather and help record, collect, share and analyse data amongst brewers.

Apart from the commercial aspect of the project, there is other scientific exploration.

With so many varied set-ups for home brewing, recipes can be tweaked and processes can be changed. The project will explore a web interface to complement the current home-brewing tech eco-system.

\section{Project aim and objectives}

    \subsection{Aims}

        \begin{itemize}
            \item Improve home-brewing processes through design and development of a web application.
            \item There will be a strong focus on user experience research and use of modern development practises.
        \end{itemize}

    \subsection{Objectives}

        \begin{itemize}
            \item Research and apply an designed solution towards a better user experience
            \item Based on UX research, create an intelligent web interface connected to the brewing kit for:
                \begin{itemize}
                    \item Guiding and instructing brewer on current brew
                    \item Recording notes and actions throughout the brew process
                    \item Connecting to current brewing software seamlessly
                \end{itemize}
            \item Analyse user experience approach through structured testing and evaluate the results.
        \end{itemize}

\section{Project deliverables}

Potential deliverables include:

\begin{itemize}
    \item Web interface for a brewing process versioning system
    \item Project report
    \item PID
    \item Trello board
\end{itemize}

\section{Project constraints}

Some of the user interface work would benefit from working on a real-life working brewing kit. While Alan is a contributor to the project, his equipment, specifically the brewing kit is not always accessible for all of the design and development of the web interface. This will mean assumptions must be made and some simulations of the running environment where needed.

\section{Project approach}

I am lacking a lot of knowledge and skills in brewing, this will take some learning. I have contacted other members in the University who have research interests in brewing to learn more.

I am approaching the development with a sprint based model in mind. Simply prioritising which work needs to be tackled in each sprint to reach the goal at hand. This goal may be more research driven or more technically driven depending on the stage of work.

\section{Facilities and resources}

No specialist equipment will be needed from the School of Computing as most of the work can be done on standard PC set-ups.

Research materials from the library will be useful for research.

For the web interface due to the nature of the application of the software it will require real-time technologicals such as an event-driven web scale server (Node.Js) and websockets (through Socket.IO).

Most of the software used will be free to use and so will not require funding. However any of the hardware to interface with the brewing kit would have to be purchased.

\section{Log of risks}

The main risk is the project requires a large amount of hours so any other concerns such as contract work or other course work will lessen the amount of hours.

\textit{\textbf{Risks table goes here}}


\section{Starting point for research}

Through talking to the client more areas of research will be made available.

In terms of the hardware two good places to look are the Open ArdBir and BrewPi projects, both specifically for home brewing.

Finding resources based on user experience in similar areas (machinery, cooking) will make sense.

\section{Breakdown of tasks}

This my initial tasks to start the project:

\begin{itemize}
    \item Research home brewing
        \begin{itemize}
            \item Review previous materials
            \item Talk to Dr Jeremy Mills
        \end{itemize}
    \item Research and choose the technology stack
        \begin{itemize}
            \item For backend versioning system
            \begin{itemize}
                \item Graph databases?
                \item CouchDB
                \item Git
            \end{itemize}
        \end{itemize}
        \begin{itemize}
            \item For the fronted web interface
            \begin{itemize}
                \item React
                \item Node.js (Express, Socket.IO)
            \end{itemize}
        \end{itemize}
\end{itemize}

\section{Project plan}

https://trello.com/b/gnePL4hX/brewing-project

I will be using Trello to organise my tasks, at each sprint I will work with my supervisor and Alan (where needed) to work out what needs to be put in the sprint. There will also be a backlog of tasks (such as research, development) which will allow a overview of the project easily. This naturally will give an idea of future work for the project too.

\section{Legal, ethical, professional and social issues}

There are no initial pressing legal/ethical/professional/social issues that may impose constraints on the project.

The project has been approved on the ethics website under the same name.

\chapter{Ethics Certificate}

% This is broken on the sharelatex pdf viewer but works on the chrome one (commented part works on built in)
\begin{center}
    % \includepdf[pages=-,scale=1.8,pagecommand={},offset=300 -300]{appendices/ethics-certificate.pdf}
    \includepdf[pages=-,pagecommand={},scale=.8,offset=0 50]{appendices/ethics-certificate.pdf}
\end{center}
